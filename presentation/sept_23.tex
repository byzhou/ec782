% Copyright 2007 by Till Tantau
%
% This file may be distributed and/or modified
%
% 1. under the LaTeX Project Public License and/or
% 2. under the GNU Public License.
%
% See the file doc/licenses/LICENSE for more details.



\documentclass{beamer}

%
% DO NOT USE THIS FILE AS A TEMPLATE FOR YOUR OWN TALKS�!!
%
% Use a file in the directory solutions instead.
% They are much better suited.
%


% Setup appearance:

\usetheme{Darmstadt}
%\usetheme{CambridgeUS}
%\usetheme{Berkeley}
%\usetheme{Hannover}
\usefonttheme[onlylarge]{structurebold}
\setbeamerfont*{frametitle}{size=\normalsize,series=\bfseries}
\setbeamertemplate{navigation symbols}{}
\setbeamertemplate{footline}{
    
    \leavevmode%
    \hbox{%
    \begin{beamercolorbox}[wd=.4\paperwidth,ht=2.25ex,dp=1ex,center]{author in head/foot}%
        \usebeamerfont{author in head/foot}\insertshortauthor
    \end{beamercolorbox}%

    \begin{beamercolorbox}[wd=.6\paperwidth,ht=2.25ex,dp=1ex,center]{title in head/foot}%
        \usebeamerfont{title in head/foot}\insertshorttitle\hspace*{3em}
        \insertframenumber{} / \inserttotalframenumber\hspace*{1ex}
    \end{beamercolorbox}}%
    \vskip0pt% 
    
    
}

% Standard packages

\usepackage[english]{babel}
\usepackage[latin1]{inputenc}
\usepackage{times}
\usepackage[T1]{fontenc}
\usepackage{graphicx}
%\usepackage{natbib}
\usepackage{enumerate}



% Setup TikZ

\usepackage{tikz}
\usetikzlibrary{arrows}
\tikzstyle{block}=[draw opacity=0.7,line width=1.4cm]


% Author, Title, etc.

\title[ Substrate-Integrated Circuits ] 
{%
Substrate-Integrated Millimeter-Wave and Terahertz Antenna Technology
  %
}

\author[Boyou Zhou]
{
  Ke Wu, Fellow IEEE\inst{1}\\
  Yu Jian Cheng, Member IEEE\inst{2}\\
  Wei Hong, Fellow IEEE\inst{4}\\
  \textit{Reviewed by Boyou Zhou\inst{*}}
}

\institute[T�bingen and others]
{
  \inst{1}
  University of Montreal, Montreal, QC, Canada\\
  \inst{2}
  Univerisity of Electronic Science and Technology, Chengdu, China\\
  \inst{3}
  INRS-EMT Montreal, QC, Canada\\
  \inst{4}
  Southeast University, Nanjing, China\\
  \inst{*}
  Boston Univeristy, MA
}

\date[fall 2014]
{EC782, \today}



% The main document

\begin{document}

\begin{frame}
  \titlepage
\end{frame}

\begin{frame}{Outline}
  \tableofcontents
\end{frame}

\AtBeginSubsection{
    \frame<beamer>{ 
    \frametitle{Outline}   
    \tableofcontents[currentsection,currentsubsection]
    }
}


\section{Introduction}
\subsection{Introduction to Terahertz-High-Frequency}

\begin{frame}{Electromagnetic Frequency}

    \begin{figure}[p]
        \includegraphics[width=4in]{../doc_image/presentation/radio_freq.png}
        \caption{Spectrum of electromagnetic waves}
        \label{Spectrum of electromagnetic waves}
    \end{figure}

    \begin{itemize}
        \item  As the technology develops, the frequency goes higher and higher.
        \item  Correspond to higher frequency, the wavelength gets shorter.
    \end{itemize}

\end{frame}

\begin{frame}{Why Terahertz Frequency?}
\pause
    \begin{block}{Photonics Perspective}
        Bridging the gap of well-perceived technology between electronics and photonics
    \end{block}
\pause
    \begin{block}{A Lot to Explore}
        Holds a number of unknown secrets, features, and promises
    \end{block}
\pause
    \begin{block}{Propagation Lost in Atmosphere}
        Propagation loss through atmosphere over millimeter-wave and terahertz frequency 
            bands goes through wild upsidedown ripple phenomena
    \end{block}
\end{frame}

\subsection{Applications in Terahertz-High-Frequency}

\begin{frame}{What is the applications of Terahertz-High-Frequency?}

    \begin{figure}[p]
        \includegraphics[width=1in]{../doc_image/presentation/medical_imaging.jpg} ~
        \includegraphics[width=1in]{../doc_image/presentation/security.jpg}~
        \includegraphics[width=1in]{../doc_image/presentation/communication.jpg}
        \caption{Various Possible Applications for THF}
        \label{Apps for THF}
    \end{figure}

    \begin{itemize}
\pause
        \item \textbf{Medical Imaging} Non-ionizing radiation provides high-resolution and
            harmless observation tools.
\pause
        \item \textbf{Security} High enough to penetrate fabrics and plastics, mainly 
            targeting sensing concealed weapon.
\pause
        \item \textbf{Communication} OF COURSE, that is what we are talking about. 
    \end{itemize}

\end{frame}

\section{Substrate-Integrated Circuits}

\subsection{Wire Connections On Board}

\begin{frame}{Electromagnetic Frequency}

    \begin{figure}[p]
        \includegraphics[width=1in]{../doc_image/presentation/wire_connection_on_board.png} ~
        \includegraphics[width=1in]{../doc_image/presentation/microstrip_line.png}
        \caption{Wire Connections on Board}
        \label{Wire Connections on Board}
    \end{figure}

    \begin{itemize}
        \item Exhibit high field/current singularities over open strip
        edges and also high transmission loss because of dielectric,
        conductor (surface roughness), and radiation losses, thus
        low Q-factor;
\pause
        \item Design and implementation of such planar
        structures are usually subject to complicated packaging
        issue, crosstalk, direct current (dc) grounding, and mode
        conversion;
    \end{itemize}
\end{frame}

\begin{frame}{Limits in Terahertz-high-frequency devices}

\pause
    \begin{itemize}
        \item Power handling (more pronounced over
        microwave and millimeter-wave frequency ranges) and
        thermal management become tedious and difficult;
\pause
        \item Fabrication
        tolerance and impedance control are difficult to meet over
        millimeter-wave and terahertz frequency range .
    \end{itemize}

\end{frame}

\subsection{Substrate-Integrated Waveguid}
\begin{frame}{Substrate-Integrated Waveguid}
    
    \begin{figure}[p]
        \includegraphics[width=1in]{../doc_image/presentation/SIW.png}~
        \includegraphics[width=1in]{../doc_image/presentation/siw2.png}~
        \includegraphics[width=1in]{../doc_image/presentation/siw3.png}
        \caption{Substrate Integrated Waveguide}
        \label{Substrate Integrated Waveguide}
    \end{figure}

    \begin{itemize}
        \item Basic planar structure as microstrip
        \item Vias on both sides with metallized inner surface
        \item Microstrip connections as matched to $50\Omega$ impedance
    \end{itemize}

\end{frame}

\begin{frame}{Substrate-Integrated Image Guide}
    
    \begin{figure}[p]
        \includegraphics[width=1.5in]{../doc_image/presentation/SIIG.PNG}
        \caption{Substrate Integrated Image Guide}
        \label{Substrate Integrated Image Guide}
    \end{figure}

    \begin{itemize}
\pause
        \item A conventional slab-image guide on
        the metallic plate from a planar substrate of certain
        thickness;
\pause
        \item This air hole perforation of
        dielectric slab at both sides of the guiding channel
        effectively lowers the permittivity or dielectric constant
    \end{itemize}

\end{frame}

\begin{frame}{Some other ways of designing the guides}
    
    \begin{figure}[p]
        \includegraphics[width=2.5in]{../doc_image/presentation/SICs.PNG}
        \caption{others}
        \label{Substrate Integrated Image Guide}
    \end{figure}

    \begin{itemize}
        \item A combination of SIW and SIIDG(Substrate Integrated Image
        Dieletric Guide)
        \item Cost-effective design and Easy for Manufactoring 
    \end{itemize}

\end{frame}

\subsection{SIC's Antenna}
\begin{frame}{Slot Array Antenna}
    
    \begin{figure}[p]
        \includegraphics[width=1.5in]{../doc_image/presentation/slot_array_antenna.png}~
        \includegraphics[width=1.5in]{../doc_image/presentation/slot_array_antenna2.PNG}
        \caption{Slot Array Antenna}
        \label{Slot Array Antenna}
    \end{figure}

    \begin{itemize}
        \item End feed type post wall waveguide fed parallel plate slot array
        \item Considering the polarization and bandwith of the antenna, the slot arrays 
        are built on SIWs with different resonant freqeuncy.
    \end{itemize}

\end{frame}

\begin{frame}{Horn Antenna}

    \begin{figure}[p]
        \includegraphics[width=2.0in]{../doc_image/presentation/horn.PNG}
        \caption{Horn Antenna}
        \label{Horn Antenna}
    \end{figure}

    \begin{itemize}
        \item One half SIW guide for the horn antenna design.
        \item Difficult to have the impedance match when the frequency is low.
    \end{itemize}

\end{frame}

\subsection{Beamforming Design}
\begin{frame}{Beamforming}
 
     \begin{figure}[p]
        \includegraphics[width=2.0in]{../doc_image/presentation/bfn.PNG}
        \caption{Beamforming Network}
        \label{Beamforming Network}
    \end{figure}

    \begin{itemize}
        \item Multibeam antenna and multiple antennas are two
        efficient methods for combating the multipath fading
        and increasing the channel capacity in radio-frequency
        (RF) transceivers and microwave systems.
        \item They used to be heavy, expensive and hard to fabricate. But
        with SIW based BFN, it is much easier to build.
    \end{itemize}
   
\end{frame}


\section{CAD Design Basic Tool Flow}
\subsection{CAD instead of Solving Equations}
\begin{frame}{Why CAD Design}
 
    \begin{figure}[p]
        \includegraphics[width=1.5in]{../doc_image/presentation/maxwell.png} ~
        \includegraphics[width=1.5in]{../doc_image/presentation/HFSS_cad.jpg}
        \caption{HFSS Demo}
        \label{HFSS Demo}
    \end{figure}

\end{frame}
    

\begin{frame}{Work Flow of SIW}
    \begin{columns}
        \column{0.5\textwidth}
\pause
            \begin{block}{Step One}
               Determine functionality of each block and specs.
            \end{block}
\pause
            \begin{block}{Step Two}
               Use the specs to design a conventional waveguide.
            \end{block}
\pause
            \begin{block}{Step Three}
               Apply the formula to the waveguide to determine the width and 
               height of the SIW and the impedance matching connection.
            \end{block}
        \column{0.5\textwidth}
\pause
            \begin{block}{Step Four}
               Draw the schematics in the HFSS.
            \end{block}
\pause
            \begin{block}{Step Five}
               Simulate the entire block to optimize the width and the connection.
            \end{block}
\pause
            \begin{block}{Step Six}
               Put the block to the entire design.
            \end{block}
    \end{columns}
 
\end{frame}
 
\section*{Summary}

\begin{frame}{Performance}
    
    \begin{figure}[p]
        \includegraphics[width=2.5in]{../doc_image/presentation/specs.PNG}
        \caption{Specs comparison}
        \label{Specs comparison}
    \end{figure}

    \begin{itemize}
        \item I did not find the comparison between the microstrip and SIW. 
        \item But I believe it's much better than microstrip when it goes to high 
            frequency.
    \end{itemize}

\end{frame}

\begin{frame}{References}
\fontsize{6}{7.2}\selectfont
        \begin{enumerate}[1]
            \item \textit{
            Ke Wu 
            Poly-Grames Res. Center, Ecole Polytech., Montreal, QC, Canada 
            Yu Jian Cheng ; Djerafi, T. ; Wei Hong
            \emph{Substrate-Integrated Millimeter-Wave and Terahertz Antenna Technology}
            Proceedings of the IEEE  (Volume:100 ,  Issue: 7 )}
            \item D. Stephens, P. Young, and I. Robertson,
            \emph{BMillimeter-wave substrate integrated
            waveguides and filters in photoimageable
            thick-film technology, IEEE Trans. Microw.
            Theory Tech., vol. 53, no. 12, pp. 3832�3838,
            Dec. 2005. }
            \item Y. J. Cheng, K. Wu, and W. Hong, 
            \emph{BPower
            handling capability of substrate integrated
            waveguide interconnects and related
            transmission line systems, IEEE Trans.
            Adv. Packag., vol. 31, no. 4, pp. 900�909,
            Nov. 2008.}
            \item A. Patrovsky and K. Wu, 
            \emph{BSubstrate integrated
            image guide (SIIG)VA planar dielectric
            waveguide technology for millimeter-wave
            applications, IEEE Trans. Microw. Theory
            Tech., vol. 54, no. 6, pt. 2, pp. 2872�2879,
            Jun. 2006.}
            \item Blumlein, Alan (1938-03-07), 
            \emph{"Improvements in or relating to high 
            frequency electrical conductors or radiators",
            British patent no. 515684}
            \item Ke Wu 
            Dept. of Electr. Eng., Ecole Polytech. de Montreal, Que., Canada 
            Deslandes, D. ; Cassivi, Y.
            \emph{
            The substrate integrated circuits - a new concept for 
            high-frequency electronics and optoelectronics
            Telecommunications in Modern Satellite, Cable and Broadcasting Service, 2003. 
            TELSIKS 2003. 6th International Conference on  (Volume:1 )
            1-3 Oct. 2003}

        \end{enumerate}
\end{frame}

\end{document}


