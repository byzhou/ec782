% Copyright 2007 by Till Tantau
%
% This file may be distributed and/or modified
%
% 1. under the LaTeX Project Public License and/or
% 2. under the GNU Public License.
%
% See the file doc/licenses/LICENSE for more details.



\documentclass{beamer}

%
% DO NOT USE THIS FILE AS A TEMPLATE FOR YOUR OWN TALKS�!!
%
% Use a file in the directory solutions instead.
% They are much better suited.
%


% Setup appearance:

\usetheme{Darmstadt}
\usefonttheme[onlylarge]{structurebold}
\setbeamerfont*{frametitle}{size=\normalsize,series=\bfseries}
\setbeamertemplate{navigation symbols}{}


% Standard packages

\usepackage[english]{babel}
\usepackage[latin1]{inputenc}
\usepackage{times}
\usepackage[T1]{fontenc}
\usepackage{graphicx}


% Setup TikZ

\usepackage{tikz}
\usetikzlibrary{arrows}
\tikzstyle{block}=[draw opacity=0.7,line width=1.4cm]


% Author, Title, etc.

\title[ Substrate-Integrated Millimeter-Wave] 
{%
Substrate-Integrated Millimeter-Wave and Terahertz Antenna Technology
  %
}

\author[Boyou Zhou]
{
  Boyou Zhou\inst{*}
}

\institute[T�bingen and others]
{
  \inst{*}%
  Boston University, Boston, MA
}

\date[fall 2014]
{EC782, \today}



% The main document

\begin{document}

\begin{frame}
  \titlepage
\end{frame}

\begin{frame}{Outline}
  \tableofcontents
\end{frame}

\AtBeginSubsection{
    \frame<beamer>{ 
    \frametitle{Outline}   
    \tableofcontents[currentsection,currentsubsection]
    }
}


\section{Introduction}
\subsection{Introduction to Ultra High Frequency}

\begin{frame}{Electromagnetic Frequency}

    \begin{figure}[p]
        \includegraphics[width=4in]{../doc_image/presentation/radio_freq.png}
        \caption{Spectrum of electromagnetic waves}
        \label{Spectrum of electromagnetic waves}
    \end{figure}

    \begin{itemize}
        \item  As the technology develops, the frequency goes higher and higher.
        \item  Correspond to higher frequency, the wavelength gets shorter.
    \end{itemize}

\end{frame}

\begin{frame}{Why higher frequency?}

    \begin{itemize}
        \item Bridging the gap of well-perceived technology between electronics and photonics
        \item Holds a number of unknown secrets, features, and promises
        \item Propagation loss through atmosphere over millimeter-wave and terahertz frequency 
                bands goes through wild upsidedown ripple phenomena
    \end{itemize}

\end{frame}

\subsection{Limits in Building Ultra-high-frequency devices}
\begin{frame}{Is that Hard to build Ultra-high-frequency devices?}

    \begin{itemize}
        \item Signals are easily radiated, crosstalked
        \item Higher Q losses
        \item Fabrication Issues
    \end{itemize}

\end{frame}

\section{Substrate-Integrated Circuits}

\subsection{Wire Connection}

\begin{frame}{Electromagnetic Frequency}

    \begin{figure}[p]
        \includegraphics[width=2in]{../doc_image/presentation/wire_connection_on_board.png}\\
        \includegraphics[width=2in]{../doc_image/presentation/microstrip_line.png}
        \caption{Wire Connection on Board}
        \label{Wire Connection on Board}
    \end{figure}

\end{frame}

\begin{frame}{Limits in Ultra-high-frequency devices}
\pause
    \begin{itemize}
        \item Exhibit high field/current singularities over open strip
        edges and also high transmission loss because of dielectric,
        conductor (surface roughness), and radiation losses, thus
        low Q-factor;
\pause
        \item Design and implementation of such planar
        structures are usually subject to complicated packaging
        issue, crosstalk, direct current (dc) grounding, and mode
        conversion;
\pause
        \item Power handling (more pronounced over
        microwave and millimeter-wave frequency ranges) and
        thermal management become tedious and difficult;
\pause
        \item fabrication
        tolerance and impedance control are difficult to meet over
        millimeter-wave and terahertz frequency range .
    \end{itemize}

\end{frame}

\subsection{Substrate-Integrated Waveguid}
\begin{frame}{Substrate-Integrated Waveguid}
    
    \begin{figure}[p]
        \includegraphics[width=1in]{../doc_image/presentation/SIW.png}~
        \includegraphics[width=1in]{../doc_image/presentation/siw2.png}~
        \includegraphics[width=1in]{../doc_image/presentation/siw3.png}
        \caption{Substrate Integrated Waveguide}
        \label{Substrate Integrated Waveguide}
    \end{figure}

    \begin{itemize}
        \item Basic planar structure as microstrip
        \item Vias on both sides with metallized inner surface
        \item Microstrip connections as matched to $50\Omega$ impedance
    \end{itemize}

\end{frame}

\begin{frame}{Substrate-Integrated Image Guide}
    
    \begin{figure}[p]
        \includegraphics[width=2.5in]{../doc_image/presentation/SICs.PNG}
        \caption{Substrate Integrated Image Guide}
        \label{Substrate Integrated Image Guide}
    \end{figure}

    \begin{itemize}
        \item A conventional slab-image guide on
        the metallic plate from a planar substrate of certain
        thickness;
        \item This air hole perforation of
        dielectric slab at both sides of the guiding channel
        effectively lowers the permittivity or dielectric constant
    \end{itemize}

\end{frame}

\begin{frame}{Substrate-Integrated Image Guide}
    
    \begin{figure}[p]
        \includegraphics[width=2.5in]{../doc_image/presentation/SICs.PNG}
        \caption{others}
        \label{Substrate Integrated Image Guide}
    \end{figure}

    \begin{itemize}
        \item A combination of several techniques
        \item Based on the needs for dielectric constant and less loss
    \end{itemize}

\end{frame}



\section*{Summary}

\begin{frame}{Summary}
    
    \begin{figure}[p]
        \includegraphics[width=1in]{../doc_image/presentation/summary.png}~
        \caption{Examples of SICs}
        \label{Examples of SICs}
    \end{figure}

    \begin{itemize}
        \item Much less loss in the Ultra-high-frequency
        \item Vias on both sides with metallized inner surface
        \item Microstrip connections as matched to $50\Omega$ impedance
    \end{itemize}

\end{frame}
\end{document}


