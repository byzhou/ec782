% Copyright 2007 by Till Tantau
%
% This file may be distributed and/or modified
%
% 1. under the LaTeX Project Public License and/or
% 2. under the GNU Public License.
%
% See the file doc/licenses/LICENSE for more details.



\documentclass{beamer}

%
% DO NOT USE THIS FILE AS A TEMPLATE FOR YOUR OWN TALKS�!!
%
% Use a file in the directory solutions instead.
% They are much better suited.
%


% Setup appearance:

\usetheme{Darmstadt}
%\usetheme{CambridgeUS}
%\usetheme{Berkeley}
%\usetheme{Hannover}
\usefonttheme[onlylarge]{structurebold}
\setbeamerfont*{frametitle}{size=\normalsize,series=\bfseries}
\setbeamertemplate{navigation symbols}{}
\setbeamercovered{transparent=30}
\setbeamertemplate{footline}{
    
    \leavevmode%
    \hbox{%
    \begin{beamercolorbox}[wd=.4\paperwidth,ht=2.25ex,dp=1ex,center]{author in head/foot}%
        \usebeamerfont{author in head/foot}\insertshortauthor
    \end{beamercolorbox}%

    \begin{beamercolorbox}[wd=.6\paperwidth,ht=2.25ex,dp=1ex,center]{title in head/foot}%
        \usebeamerfont{title in head/foot}\insertshorttitle\hspace*{3em}
        \insertframenumber{} / \inserttotalframenumber\hspace*{1ex}
    \end{beamercolorbox}}%
    \vskip0pt% 
    
    
}

% Standard packages

\usepackage[english]{babel}
\usepackage[latin1]{inputenc}
\usepackage{times}
\usepackage[T1]{fontenc}
\usepackage{graphicx}
%\usepackage{natbib}
\usepackage{enumerate}



% Setup TikZ

\usepackage{tikz}
\usetikzlibrary{arrows}
\tikzstyle{block}=[draw opacity=0.7,line width=1.4cm]

%specify the image path
\graphicspath{{../../img/presentation/}}

% Author, Title, etc.

\title[ New Transceiver Architecture ] 
{%
A New Transceiver Architecture for the 60-GHz Band
  %
}

\author[Boyou Zhou]
{
  Ali Parsa, Member IEEE\inst{1}\\
  Behzad Razavi, Fellow IEEE\inst{1}\\
  \textit{Reviewed by Boyou Zhou\inst{*}}
}

\institute[T�bingen and others]
{
  \inst{1}
  University of California at Los Angeles, CA\\
  \inst{*}
  Boston Univeristy, MA
}

\date[fall 2014]
{EC782, \today}



% The main document

\begin{document}

\begin{frame}
  \titlepage
\end{frame}

\begin{frame}{Outline}
  \tableofcontents
\end{frame}

\AtBeginSubsection{
    \frame<beamer>{ 
    \frametitle{Outline}   
    \tableofcontents[currentsection,currentsubsection]
    }
}

\section{Introduction}
\subsection{60GHz Communication Application}
\begin{frame}{60GHz Commmunication Application}
    \begin{itemize}
        \item [*] 60GHz communication refres to frequencies between $57 GHz$ and
        $66 GHz$.
\pause
        \item [*] FCC has allocated as licence-free operation. 
\pause
        \item [*] Energy propagation in the 60 GHz band has unique
        characteristics and benefits.
\pause
        \item [*] More information can be transmitted and received.
    \end{itemize}
\end{frame}

\subsection{60GHz Technology Design Difficulties}
\begin{frame}{Design Difficulties}
    \begin{itemize}
        \item \textit{Generic Challenges} The system design requires high
        frequency operations and low-noise design.
\pause
        \item \textit{LO Design} This paper mainly focus on LO(\textit{Local
        Oscillation}) design.
\pause
            \begin{itemize}
                \item [-] \textbf{Generation} The system needs quadrature LO
                generation.
                \item [-] \textbf{Division} For local oscillation frequency, it
                may requires frequency division.
                \item [-] \textbf{Distribution} The local oscillation signal
                needs to distributed for different mixers.
            \end{itemize}
    \end{itemize}
\end{frame}

\section{Compararison of Architectures}
\subsection{Traditional Architectures}
\begin{frame}{Design of High Frequency System Local Oscillators}
    \begin{block}{High Q Issues}
        The generation of quadrature signal is a very difficult task for system
        working at $60GHz$, since the Q values for inductors and varactors begin
        to saturate at high frequencies.
    \end{block}

\pause
    \begin{block}{Frequency Doubler Issues}
        Applying a frequency doubler in the system may leads to some other
        problems.
    \end{block}

\end{frame}

\subsection{Problems with Traditional Architectures}

\begin{frame}{Designs of the Transceivers}
    \begin{figure}[p]
        \includegraphics[width=4in]{preArchitecture.PNG}
        \caption{Traditional Architecture of Transceiver}
    \end{figure}
    \begin{itemize}
\pause
        \item The first figure shows a dummy version of how the basic
        architecture works. It requires high Q for the lumped components.
\pause
        \item The second figure has these following drawbacks.
    \end{itemize}
\end{frame}

\begin{frame}{The Third Kind of Design}
    \begin{itemize}
\pause
        \item Frequency Doubler Issues
            \begin{itemize}
                \item [*] CMOS doublers tend to be quite lossy at these
                frequencies, raising noise floor, necessitiating
                post-amplification to achieve sufficient swings.
                \item [*] Typical design does not provide quadrature outputs
                \item [*] Since it requires additional inductors, the
                distribution can be very hard.
            \end{itemize}
\pause
        \item The third kind of design illustrates a general way of having the
        first stage $Nf_{osc}$ and the second stage $f_{osc}/M$ downconverting.
        Thus the input signal and local oscillation signal frequencies should
        satisfy the equ.~\ref{osc&in}.
        \begin{equation}
            \frac{f_{osc}}{M} = f_{in} - Nf_{osc}
            \label{osc&in}
        \end{equation}
    \end{itemize}
\end{frame}

\begin{frame}
        Equ.~\ref{oscVSin} also shows that theis architecture should deal with 
        \begin{equation}
            f_{osc} = f_{in} \times \frac{M}{MN+1}
            \label{oscVSin}
        \end{equation}
\pause
        \begin{itemize}
            \item [-] \textbf{Loss of the Frequency Multiplier}
\pause
            \item [-] \textbf{Problem of Image Rejection}
        \end{itemize}
\end{frame}

\section{Receiver Design}
\subsection{I/Q Gain Mismatch}
\begin{frame}{Designs of the Transceivers}
    \begin{figure}[p]
        \includegraphics[width=4in]{halfrftrans.png}
        \caption{This Paper's Design}
    \end{figure}

    $Re{x_{BB}(t)e^{+j\omega_{RF}t}}$ has been modulated by $cos \omega_{LO}t +
    \alpha cos 3\omega_{LO}t$ caused by LO's third harmonics modulation.
    \begin{equation}
        x_{IF}(t) = Re\{x_{BB}(t)e^{+j\omega_{IF}t}\} + Re\{\alpha
        x_{BB}^*(t)e^{+j\omega_{IF}t}\}
    \end{equation}
\end{frame}

\begin{frame}{I/Q Gain Mismatch}
    Thus the signal will have an I/Q gain mismatch. 
        \begin{equation}
            \begin{split}
            x_{BB,I}(t) & = Re\{x_{BB}(t) + \alpha x_{BB}^*(t)\} \\
                        & = ( 1 + \alpha ) Re\{x_{BB}(t)\}
            \end{split}
        \end{equation}

        \begin{equation}
            \begin{split}
            x_{BB,Q}(t) & = Im\{x_{BB}(t) + \alpha x_{BB}^*(t)\} \\
                        & = ( 1 - \alpha ) Im\{x_{BB}(t)\}
            \end{split}
        \end{equation}
\end{frame}

\begin{frame}{The Third Harmonics from LO}
    \begin{figure}[p]
        \includegraphics[width=2in]{16QAM.png}
        \caption{Impacts of I/Q gain mismatch on 16QAM}
    \end{figure}
\end{frame}

\subsection{Cancelling I/Q Gain Mismatch}
\begin{frame}{Designs of the Receivers}
    \begin{figure}[p]
        \includegraphics[width=2in]{halfrfReceiver.png}~
        \includegraphics[width=2in]{halfrfreceiverspectrum.png}
        \caption{Receiver Design}
    \end{figure}
\end{frame}

\begin{frame}{Designs of the Receivers}
    \begin{figure}[p]
        \includegraphics[width=1in]{LNA.png}~
        \includegraphics[width=3in]{polyphasefilter.png}
        \caption{Receiver Design}
    \end{figure}
\end{frame}

\begin{frame}{Designs of the Receivers}
    \begin{figure}[p]
        \includegraphics[width=4in]{receiverreal.png}
        \caption{Receiver Design}
    \end{figure}
\end{frame}


\section{Transmitter Design}
\subsection{Transmitter Design}
\begin{frame}{Designs of the Transmiiter}
    \begin{figure}[p]
        \includegraphics[width=2in]{transmitter.png}~
        \includegraphics[width=2in]{transmitterspectrum.png}
        \caption{Transmitter Design}
    \end{figure}
\end{frame}

\section{Validation}
\subsection{Validataion of the Design}
\begin{frame}{Validation of the Design}
    \begin{figure}[p]
        \includegraphics[width=4in]{layout.png}
        \caption{Layout}
    \end{figure}
\end{frame}

\section{Specs}
\subsection{Results}
\begin{frame}{Gain and Noise Figure of Receiver}
    \begin{figure}[p]
        \includegraphics[width=4in]{recgainandnf.png}
        \caption{Gain and Noise Figure of Receiver}
    \end{figure}
\end{frame}

\begin{frame}{Gain and Noise Figure of Transmitter}
    \begin{figure}[p]
        \includegraphics[width=4in]{transgainandnf.png}
        \caption{Gain and Noise Figure of Transmitter}
    \end{figure}
\end{frame}

\begin{frame}{Thanks}
        \textbf{Any Questions?}
\end{frame}

\end{document}


