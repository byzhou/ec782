\documentclass[]{article}
\usepackage{graphicx}
\usepackage{multicol}

\parindent=0pt
\usepackage[margin=0.5in]{geometry}

\begin{document}
\pagestyle{empty}
{\large\textbf{EC782 Lab3 Quadrature VCO}}

%\begin{multicols}{2}

    \begin{itemize}
        \item[*] Originally Created on Sept 11, 2014
        \item[*] Modified on \today
        \item[*] \textit {Boyou Zhou}\\
                 8 St Mary's St, PHO 340, Boston, MA 02215\\
                 Email: bobzhou@bu.edu, Phone: 617-678-8480
                    
    \end{itemize}

%    \columnbreak
%
%    \begin{itemize}
%        \item[ ]
%        \item[ ]
%        \item[*] \textit {Boyou Zhou}\\
%                 8 St Mary's St, PHO 340, Boston, MA 02215\\
%                 Email: bobzhou@bu.edu, Phone: 617-678-8480
%    \end{itemize}

%\end{multicols}

\rule[-0.1cm]{7.5in}{0.01cm}\\
\\
\indent		             
\section {Summary of Results}

\begin{center}
    \begin{tabular}{c c}
        Parameter & values \\ \\ \hline \\
        DC current & 13.08mA \\
        Power Comsumption & 19.62 \\
        S22 Impedance & $50.48\Omega$ \\
        Tunning Range & $5.18 - 4.78 GHz$ \\
        Phase Noise (10kHz) & $-39.61dBc/Hz$ \\
        Phase Noise (10MHz) & $-130dBc/Hz$


    \end{tabular}
\end{center}

\begin{center}
    Table 1: Main Parameters 
\end{center}

\section{VCO Architecture}
The entire design is consists of a quadrature VCO and two super gain amplifier.
The super gain amplifier used a current mirror to copy the current to the output
node. And there is a positive feedback loop in the amplifier to increase the
input current node in the input of the current mirror.

\end{document}

