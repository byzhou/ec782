\documentclass[]{article}
\usepackage{graphicx}
\usepackage{multicol}

\parindent=0pt
\usepackage[margin=0.5in]{geometry}

\begin{document}
\pagestyle{empty}
{\large\textbf{EC782 Lab2 Noise Figure Measurement Report}}

\begin{multicols}{2}

    \begin{itemize}
        \item[*] Originally Created on Sept 6, 2014
        \item[*] Modified on \today
        \item[*] Author info: \textit{Yalun Zhang}\\ 
                 Email:yalunz@bu.edu
                    
    \end{itemize}

    \columnbreak

    \begin{itemize}
        \item[ ]
        \item[ ]
        \item[*] \textit {Boyou Zhou}\\
                 8 St Mary's St, PHO 340, Boston, MA 02215\\
                 Email: bobzhou@bu.edu, Phone: 617-678-8480
    \end{itemize}

\end{multicols}

\rule[-0.1cm]{7.5in}{0.01cm}\\
\\
\indent		             
\section {Summary of Results}

\begin{center}
    \begin{tabular}{c c}
        Parameter & values \\ \\ \hline \\
        Resolution Bandwidth & 10kHz \\
        Centre Frequency & 2GHz \\
        Temperature & 290K \\
        Noise Figure & 19.54dB

    \end{tabular}
\end{center}

\begin{center}
    Table 1: Main Parameters 
\end{center}

\section{Experiment Procesure}

According to the Thomas Lee's book, \textit{the Good Old Method} , noise figure can
be measured by the equation[\ref{noise figure}]. Because the noise that we are 
measuring, $N_{measured}$ , is too low to be observed through the spectrum analyzer.
It it necessary to push the noise higher than the noise floor so that it can be measured.
In the experiment, our group cascade the amplifier with two other low noise amplifier.
The input signal is fed into the \textit{DUT}, the amplifier to be measured, and then 
connect the \textit{DUT} with the two cascaded amplifiers.

\begin{equation}
    Noise Figure = \frac{N_{measured}}{kBT_{0}G_{total}}
    \label{noise figure}
\end{equation}

In order to know how much gain we have applied into the noise signal, we need to first 
measure the total gain from the cascaded last two amplifiers. But before the measurements,
the entire system needs to be configured. Thus, we need to first measure how much loss 
or gain can be get from the function generator to the spectrum analyzer.

\begin{center}
    \begin{tabular}{c c}
        Parameter & values \\ \\ \hline \\
        Input Signal \textit{measured from function generator} & -20dBm \\
        Output Signal \textit{measured from spectrum analyzer} & -16.6dBm
    \end{tabular}
\end{center}

\begin{center}
    Table 2: Signal Amplifier Configuration
\end{center}

After configured by measuring the signal from function generator to the spectrum analyzer,
the total gain including the last two stages of amplifier need to be measured. Since cascaded
amplifiers' noise figure is mainly determined by the first stage's amplifier, equation
[\ref{cascading amplifiers}] , we can estimate the noise figure of the entire system should
be the same as the first stage's noise figure.

\begin{equation}
    Noise Figure = F_{1} + \frac{F_{2} - 1}{G_{1}} + \frac{F_{3} - 1}{G_{1}G_{2}}
    \label{cascading amplifiers}
\end{equation}

The total system gain is measured, by measuring each single stage, and the data is 
in the table[3] below. \\ \\ \\

\begin{center}
    \begin{tabular}{c c c c}
        Parameter & input & output & gain\\ \\ \hline \\
        5916 Gain  & -20dBm & -5.419dBm & 14.6dB\\
        2522 Gain  & -20dBm & 0.001dBm & 20dB\\
        2531 Gain  & -20dBm & 3.974dBm & 24dB
    \end{tabular}
\end{center}

\begin{center}
    Table 3: Amplifier Gain
\end{center}

By applying the equation[\ref{noise figure}] , the noise figure of \textit{DUT} is $19.54dB$.

\section{Pictures}

\begin{figure}[p]
    \begin{center}
            \includegraphics[width=5in]{../../img/lab2/lab2_1.png}~
            \caption{Noise Measurement}
    \end{center}
\end{figure}

\begin{figure}[p]
    \begin{center}
            \includegraphics[width=5in]{../../img/lab2/lab2_2.png}
            \caption{Configuration}
    \end{center}
\end{figure}


\end{document}

