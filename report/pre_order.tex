\documentclass[]{article}
\usepackage{graphicx}
\usepackage{multicol}

\parindent=0pt
\usepackage[margin=0.5in]{geometry}

\begin{document}
\pagestyle{empty}
{\large\textbf{EC782 Pre-PCB-order Brief Report}}

\begin{multicols}{2}

    \begin{itemize}
        \item[*] Originally Created on Nov 3, 2014
        \item[*] Modified on \today
        \item[*] Author info: \textit{Yalun Zhang}\\ 
                 Email:yalunz@bu.edu
                    
    \end{itemize}

    \columnbreak

    \begin{itemize}
        \item[ ]
        \item[ ]
        \item[*] \textit {Boyou Zhou}\\
                 8 St Mary's St, PHO 340, Boston, MA 02215\\
                 Email: bobzhou@bu.edu, Phone: 617-678-8480
    \end{itemize}

\end{multicols}

\rule[-0.1cm]{7.5in}{0.01cm}\\
\\
\indent		             
\section {Summary of Results}

\begin{center}
    \begin{tabular}{c c}
        Parameter & Values \\ \\ \hline \\
        Total PCB area & $10mm x 10mm$ \\
        Estimated Gain & $152dB$ \\
        Estimated Noise Figure & $0.9$ \\
        Total Cost & $\$153 $ (Excluding SMA Connectors)   \\
        Total Power Consumption & $100mW$
    \end{tabular}
\end{center}

\begin{center}
    \textbf{Table} A Brief Summary of results
\end{center}

\section{Design Work Flow}
The transceiver design in the RF area is largely determined by the gain, noise
figurer, cost and other aspects. These needs to be considered in the entire
design. Usually, the link budget estimation, which determines how much designers
are going to pay for the entire design and how much gain the entire receiver is
going to have.
Thus, the entire receiver design flow follows the steps below:
    \begin{itemize}
        \item \textit{Estimation the specs of the entire design} \\
            The receiver front end will pick up the signal almost $2-3$ meters
            from the transmitter front end. Thus the signal is going to decade
            almost $40-60$dB as the signal is transmitting through the air. The
            following equation shows the estimation of receiver front-end signal
            signal power, as in the Equ.~\ref{power_est}.
            \begin{equation}
                P_{R} = \phi_{R}A_{e}=\frac{P_{T}}{4\pi R^2}A_{e}
                \label{power_est}
            \end{equation}
            $\phi_{R}$ is the free-space signal density, which follows a rule of
            quadrature loss. $A_{e}$ is the efficiency of the antenna
            (\textit{power reflection has not been taken into consideration.})
            $P_{T}$ is the radiated power from the transmitter. $R$ is the
            distance from the transmitter to the receiver. After the
            calculation, the estimation of the signal strength that can be
            picked up by the receiver is $-60dBm$.
        \item \textit{Architecture Desicision} \\
            The entire design is to down-converting the signal from $1GHz$ to
            $IF$ frequency and then the FM signal being demoduled by the
            demodulator. The possible choices of the demodulation architecture
            are mainly decided by the demodulator. Since in the datasheet of
            \textit{SA636}, the details of the application circuits are very
            detailed. Thus the demodulator from our design is \textit{SA636}. It
            is using both $I/Q$ signal in the design. Therefore, in the design,
            a-$90$-degree phase shift is needed. But this has also been designed
            by the application circuit. So the only part the designer needs to
            worry about is to downconverting the signal from $1GHZ$ to the
            frequency that $SA636$ can be demodulated. In this case, the
            frequency is $100MHz$.
            \begin{figure}[p]
                \begin{center}
                    \includegraphics[width=4in]{../../img/pre_order/architecture.png}~
                    \label{architecture}
                    \caption{Architecture of the Receiver}
                \end{center}
            \end{figure}
        \item \textit{Main Components} \\
            Once all the specs have been determined, the designer should choose
            the main components from the websites like \textit{Digikey}, to see
            whether they are available. The first part is to choose the main
            components that my affect the entire design. These components
            include LNA, mixer, demodulator.

            \begin{center}
                \begin{tabular}{c c c c c}
                    Component Catogory & Name of the Components & Part Number &
                    Manufacture & Price \\ \\ \hline \\
                    Low Noise Amplifier & ADL5521 & blabla & Analog Devices Inc.
                    & \$3 \\
                    Mixer & ADL5521 & blabla & Analog Devices Inc. & \$3 \\
                    Manufacturer & ADL5521 & blabla & Analog Devices Inc. & \$3
                    \\
                    Demodulator & ADL5521 & blabla & Analog Devices Inc. & \$3
                    \\
                \end{tabular}
            \end{center}

            \begin{center}
                \textbf{Table} Main Components List 
            \end{center}

        \item \textit{Other Components} \\
            For other \textit{unimportant} components, the priority is to make
            the layout more adjustable for the entire design. Thus the choice of
            the other components should based on both specs and the footprint
            designs. The brief list of the other components is below.

            \begin{center}
                \begin{tabular}{c c c c c}
                    Component Catogory & Name of the Components & Part Number &
                    Manufacture & Price \\ \\ \hline \\
                    Components & ADL5521 & blabla & Analog Devices Inc.
                    & \$3 \\
                    Inductors & ADL5521 & blabla & Analog Devices Inc. & \$3 \\
                    Resistors & ADL5521 & blabla & Analog Devices Inc. & \$3
                    \\
                    SMA & ADL5521 & blabla & Analog Devices Inc. & \$3 \\
                    Audio Jack & ADL5521 & blabla & Analog Devices Inc. & \$3 \\
                \end{tabular}
            \end{center}

            \begin{center}
                \textbf{Table} Other Components List 
            \end{center}

        \item \textit{Draw the Schematics and Layout} \\
            The schematic is in fig.~\ref{schematic}. And the layout is in
            fig.~\ref{layout}.
            \begin{figure}[p]
                \begin{center}
                    \includegraphics[width=4in]{../../img/pre_order/schematic.PNG}~
                    \label{schematic}
                    \caption{Schematic of the Design}
                \end{center}
            \end{figure}

            \begin{figure}[p]
                \begin{center}
                    \includegraphics[width=4in]{../../img/pre_order/layout.PNG}~
                    \label{layout}
                    \caption{Layout of the Design}
                \end{center}
            \end{figure}
            
        \item \textit{Verifying the Cost and Specs of the Entire Design} \\
            The Cost and specs are listed at the beginning of the report.

    \end{itemize}

\section{Main Components Selections}
\subsection{Low Noise Amplifier}
\subsection{Filter}
\subsection{Mixer}
\subsection{Oscillators}
\subsection{Demodulator}
\section{Other Components}
\subsection{LNA Lumped Elements Selections}
\subsection{Other Capacitors, Inductors, and Resistors}
\subsection{SMA}
\subsection{Test Points}
\subsection{Audio Jack}
\section{System Parameter Calculation}
\section{Components List}
\section{Summary}

\end{document}

