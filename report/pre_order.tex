\documentclass[]{article}
\usepackage{graphicx}
\usepackage{multicol}

\parindent=0pt
\usepackage[margin=0.5in]{geometry}

\begin{document}
\pagestyle{empty}
{\large\textbf{EC782 Pre-PCB-order Brief Report}}

\begin{multicols}{2}

    \begin{itemize}
        \item[*] Originally Created on Nov 3, 2014
        \item[*] Modified on \today
        \item[*] Author info: \textit{Yalun Zhang}\\ 
                 Email:yalunz@bu.edu
                    
    \end{itemize}

    \columnbreak

    \begin{itemize}
        \item[ ]
        \item[ ]
        \item[*] \textit {Boyou Zhou}\\
                 8 St Mary's St, PHO 340, Boston, MA 02215\\
                 Email: bobzhou@bu.edu, Phone: 617-678-8480
    \end{itemize}

\end{multicols}

\rule[-0.1cm]{7.5in}{0.01cm}\\
\\
\indent		             
\section {Summary of Results}

\begin{center}
    \begin{tabular}{c c}
        Parameter & Values \\ \\ \hline \\
        Total PCB area & $10mm \times 10mm$ \\
        Estimated Gain & $152dB$ \\
        Estimated Noise Figure & $0.9$ \\
        Total Cost & $\$153 $ (Excluding SMA Connectors)   \\
        %Total Power Consumption & $100mW$
    \end{tabular}
\end{center}

\begin{center}
    \textbf{Table} A Brief Summary of results
\end{center}

\section{Design Work Flow}
The transceiver design in the RF area is largely determined by the gain, noise
figurer, cost and other aspects. These needs to be considered in the entire
design. Usually, the link budget estimation, which determines how much designers
are going to pay for the entire design and how much gain the entire receiver is
going to have.
Thus, the entire receiver design flow follows the steps below:
    \begin{itemize}
        \item \textit{Estimation the specs of the entire design} \\
            The receiver front end will pick up the signal almost $2-3$ meters
            from the transmitter front end. Thus the signal is going to decade
            almost $40-60dB$ as the signal is transmitting through the air. The
            following equation shows the estimation of receiver front-end signal
            signal power, as in the Equ.~\ref{power_est}.
            \begin{equation}
                P_{R} = \phi_{R}A_{e}=\frac{P_{T}}{4\pi R^2}A_{e}
                \label{power_est}
            \end{equation}
            $\phi_{R}$ is the free-space signal density, which follows a rule of
            quadrature loss. $A_{e}$ is the efficiency of the antenna
            (\textit{Power reflections have not been taken into consideration.})
            $P_{T}$ is the radiated power from the transmitter. $R$ is the
            distance from the transmitter to the receiver. After the
            calculation, the estimation of the signal strength that can be
            picked up by the receiver is $-60dBm$.
        \item \textit{Architecture Desicision} \\
            The entire design is to down-converting the signal from $1GHz$ to
            $IF$ frequency and then the FM signal will be demoduled by the
            demodulator. The possible choices of the demodulation architecture
            are mainly decided by the demodulator. Since in the datasheet of
            \textit{SA636}, the application circuits are very detailed. Thus the
            demodulator from our design is \textit{SA636}. It is using both
            $I/Q$ signal in the design. Therefore, in the design, a-$90$-degree
            phase shift is needed. But this has also been designed by the
            application circuit. So the only part the designer needs to worry
            about is to downconverting the signal from $1GHZ$ to the frequency
            that $SA636$ can be demodulated. In this case, the frequency is
            $100MHz$.
            \begin{figure}[p]
                \begin{center}
                    \includegraphics[width=4in]{../../img/pre_order/architecture.png}~
                    \label{architecture}
                    \caption{Architecture of the Receiver}
                \end{center}
            \end{figure}
        \item \textit{Main Components} \\
            Once all the specs have been determined, the designer should choose
            the main components from the websites like \textit{Digikey}, to see
            whether they are available. The first part is to choose the main
            components that my affect the entire design. These components
            include LNA, mixer, demodulator.

            \begin{center}
                \begin{tabular}{c c c c c}
                    Component Catogory & Name of the Components & Part Number &
                    Manufacture & Price \\ \\ \hline \\
                    Low Noise Amplifier & ADL5521 & ADL5521ACPZ-R7CT-ND & Analog Devices Inc.
                    & \$4.08 \\
                    Mixer & UPC2757TB/58TB & UPC2757TB-E3-ACT-ND & CEL & \$2.87 \\
                    Demodulator & SA636 & 568-4327-1-ND & NXP Semiconductors &
                    \$5.54\\
                    VCO\(120MHz\) & CVCO33CL-0110-0150 & 744-1056-ND & Crystek
                    Corporation & \$23.10 \\
                    VCO\(890MHz\) & CVCO55CL-0800-0980 & 744-1196-ND & Crystek
                    Corporation & \$25.30 \\
                \end{tabular}
            \end{center}

            \begin{center}
                \textbf{Table} Main Components List 
            \end{center}

        \item \textit{Other Components} \\
            For other \textit{unimportant} components, the priority is to make
            the layout more adjustable for the entire design. Thus the choice of
            the other components should based on both specs and the footprint
            designs. The brief list of the other components is below.

        \item \textit{Draw the Schematics and Layout} \\
            The schematic is in fig.~\ref{schematic}. And the layout is in
            fig.~\ref{layout}.
            \begin{figure}[p]
                \begin{center}
                    \vspace{-0.3in}
                    \includegraphics[width=6in]{../../img/pre_order/schematic.PNG}~
                    \label{schematic}
                    \caption{Schematic of the Design}
                \end{center}
            \end{figure}

            \begin{figure}[p]
                \begin{center}
                    \includegraphics[width=6in]{../../img/pre_order/layout.PNG}~
                    \label{layout}
                    \caption{Layout of the Design}
                \end{center}
            \end{figure}
            
        \item \textit{Verifying the Cost and Specs of the Entire Design} \\
            The Cost and specs are listed at the beginning of the report.

    \end{itemize}

\section{Main Components Selections}
    \subsection{Low Noise Amplifier}
        Low Noise Amplifier is one of the most critical components in the entire
        design. Since the FM receiver performance is mainly decided by the SNR,
        which consists of signal strength and noise figure. The noise figure of
        a cascading system is mainly decided by the first several stages. In
        most of the communication system, the first several stages are consists
        of LNA and filters. In this case, we did not apply an RF filter in the
        front end, thus the filter noise figure has not been taken
        consideration, according to Equ.~\ref{cascading}.
        \begin{equation}
            Noise Figure = F_{1} + \frac{F_{2} - 1}{G_{1}} + \frac{F_{3} -
            1}{G_{1}G_{2}} + ...
            \label{cascading}
        \end{equation}
        The choice for the LNA highly depends on the noise figure and the gain.
        Since Analog Devices Inc.'s ADL5521 provides detailed information of
        application of the low noise amplifier, we choose it for our design. The
        application circuit gives detailed information of the bias circuit and
        the input impedance matching. The data sheet provides the suggestions of
        the each components value at different frequencies. The specs of ADL5521
        is listed below.

    \subsection{Filter}
        Since the RF filter is not needed in this project, the only filter that
        we need is the IF filter that can filter out the mirror signal higher
        than the local oscillary frequency. We choose the oscillator at 890MHz,
        which can also be tuned from 800MHz to 1GHz. The IF filter is at 100MHz
        and it is a passband filter. We use the custome way of filter design and
        use lumped elements to build the filter.

        \begin{center}
            \begin{tabular}{c c c}
                Spec & Value \\ \\ \hline \\
                gain & $21dB$ \\
                Noise Figure & $0.9dB$ \\
            \end{tabular}
        \end{center}

        \begin{center}
            \textbf{Table} LNA Specs
        \end{center}

    \subsection{Mixer}
        The only consideration for the mixer is the working frequency. We choose
        the mixer with a working frequency from $400MHz$ to $2GHz$. The
        conversion gain for the mixer is decent at $1GHz$ and the noise figure
        is not so good. Since the cascading system, the system noise figure is
        irrelavant with the last few stages. Because Equ.~\ref{cascading}thus it
        does not matters if the noise figure of the mixer does not matter.

        \begin{center}
            \begin{tabular}{c c c}
                Spec & Value \\ \\ \hline \\
                gain & $10.1dB$ \\
                Noise Figure & $12dB$ \\
            \end{tabular}
        \end{center}

        \begin{center}
            \textbf{Table} Mixer Specs
        \end{center}

    \subsection{Oscillators}
        The frequency we choose for the IF is $110MHz$. Thus the oscillator we
        choose can be tuned to that frequency.

        \begin{center}
            \begin{tabular}{c c c}
                Spec & Value \\ \\ \hline \\
                Frequency & $800MHz-1GHz$ \\
                Phase Noise & $110dBc/Hz$ \\
            \end{tabular}
        \end{center}

        \begin{center}
            \textbf{Table} Oscillator Specs
        \end{center}

        The Oscillator for the demodulator has the specs as below.

        \begin{center}
            \begin{tabular}{c c c}
                Spec & Value \\ \\ \hline \\
                Frequency & $110Hz$ \\
                Phase Noise & $110dBi/Hz$ \\
            \end{tabular}
        \end{center}

        \begin{center}
            \textbf{Table} Oscillator Specs
        \end{center}

    \subsection{Demodulator}
        \subsubsection{Principle}
            In the RF systems, signal frequency of the antenna received is
            usually high and narrow bandwidth. Under the current technology, we
            always use the mixer to convertdown high frequency. And then flow
            mid-frequency signal into filters, LNA and demodulator.\par
            The mixer output is processed with a multistage IF amplifier, with
            the 10.7 MHz IF passband shaping formed using ceramic filters. As
            the characteristics are quiet fit our design, we choose SA636 as our
            demodulator. The SA636 is a low-voltage high performance monolithic
            FM IF system with high-speed RSSI incorporating a mixer/oscillator,
            two limiting intermediate frequency amplifiers, quadrature detector,
            logarithmic RSSI(Received Signal Strength Indicator), voltage
            regulator, wideband data output and fast RSSI op amps. \par
            The input stage is a Gilbert cell mixer with oscillator. Typical
            mixer characteristics include a noise figure of 14 dB, conversion
            gain of 11 dB. To achieve optimum linearity of the log signal
            strength indicator, there must be a 6dBV insertion loss between the
            first and second IF stages. The signal from the second limiting
            amplifier goes to a Gilbert cell quadrature detector. One port of
            the Gilbert cell is internally driven by the IF. The other output of
            the IF is AC-coupled to a tuned quadrature network. This signal,
            which now has a 90° phase relationship to the internal signal,
            drives the other port of the multiplier cell. Overall, the IF
            section has a gain of 90 dB for operation at intermediate frequency
            at 10.7 MHz. \par
        \subsubsection{Quadrature detector}
            A quadrature demodulator for FM signals consists out of two parts: a
            frequency dependent phase shift network and a phase detector. Two
            signals are considered to be in phase quadrature when their phase
            difference is exactly 90°.\par
            A quadrature detector splits the signal, that is to be demodulated,
            up into two signals. The first signal remains unchanged and is
            applied to the input of a phase detector. The second signals is
            passed through a capacitor. The capacitor will create a 90° phase
            shift. The phase shifted signal is then applied to a frequency
            dependent phase shifter, a LC tank circuit. If the LC tank circuits
            center frequency equals the signals frequency, it imposes a further
            phase shift of 0° on the signal.  Therefore the overall phase
            difference will remain 90°.\par
            If the signals frequency does vary from the LC circuits center
            frequency, the LC tank will further shift the phase of the signal
            from the capacitor, so that the signal’s total phase shift will be
            the sum of the 90° shift, that’s imposed by the capacitor, and the
            additional positive or negative phase change that is imposed by the
            LC tank circuit. If the frequency increases, the phase shift of the
            LC circuit decreases and vice versa. Therefore, the overall phase
            difference decreases as the frequency increases and vice versa.\par

            \begin{figure}[p]
                \begin{center}
                    \includegraphics[width=4in]{../../img/pre_order/demod.PNG}~
                    \label{ADL5521}
                    \caption{The Schematic of Demodulation}
                \end{center}
            \end{figure}

        \subsubsection{SA636 Features}

            \begin{figure}[p]
                \begin{center}
                    \includegraphics[width=4in]{../../img/pre_order/pin.PNG}~
                    \label{Pins of the ADL5521}
                    \caption{The Schematic of Pins for SA636}
                \end{center}
            \end{figure}

            \begin{figure}[p]
                \begin{center}
                    \includegraphics[width=4in]{../../img/pre_order/testcircuit.PNG}~
                    \label{Testcircuits}
                    \caption{The Test Circuit for Demodulator SA636}
                \end{center}
            \end{figure}

            Fig.~\ref{Testcircuits} shows the block diagram of SA636 and Figure
            2 is the pinning information of that. 
        
\section{Other Components}
    
    \subsection{LNA Lumped Elements Selections} 
        The LNA lumped elements are chosen according to the requirements on the
        data sheet. The data sheet for ADL5521 gives a lot of detailed values
        for the value of the impedance match and the BFL.  Here is the specs for
        the ADL5521 at different frequency.

        \begin{center}
            \begin{tabular}{c c c c c}
                Frequency & Noise Figure & Gain(dB) & P1dB(dBm) & OIP3(dBm) \\ \\ \hline \\
                900MHz & 0.9 & 20.8 & 20.8 & 34.0\\
                1950MHz & 1.0 & 15.8 & 21.0 & 35.0\\
                2600MHz & 1.0 & 13.0 & 20.6 & 35.0\\
                3500MHz & 1.1 & 10.5 & 20.3 & 36.0
            \end{tabular}
        \end{center}

        \begin{center}
            \textbf{Table} LNA Specs
        \end{center}

        \begin{figure}[p]
            \begin{center}
                \includegraphics[width=4in]{../../img/pre_order/ADL5521app.PNG}~
                \label{ADL5521}
                \caption{The application of ADL5521}
            \end{center}
        \end{figure}

    \subsection{Other Capacitors, Inductors, and Resistors}
        The selection of the lumped elements are based on the whether the
        elements are easy for soldering. Thus we all choose the elements with
        the package of 0603. If 0603 package is not available, we will chose the
        0402 packages. For the input of LNA, we need to use the elements with
        higher Q value, since the impedance match needs to have a low energy
        loss due to the elements themselves.
    \subsection{SMAs and Test Points}
        The SMA connectors are for the input signal of the entire system and are
        also used for the test points of each module. The test points are
        inserted in the output of oscillators and the output of the amplifiers.
        In this case, the debugging part is for the amplifiers' impedance match
        and the oscillators tunning.
    \subsection{Audio Jack}
        The audio jack consists of three ends, sleeve, tip and ring. Sleeve is
        for the left channel of the sound signal and tip is for the right
        channel, while ring is the ground.

\section{System Parameter Calculation}
    System Gain Calculation
    \begin{equation}
        System Gain (Front end) = 20.1dB \times 3 + 10dB = 70.1dB
    \end{equation}

    System Noise Calculation
    \begin{equation}
        System Gain (Front end) = 0.9 +
        \frac{10^(\frac{0.9}{10})-1}{10^\frac{20}{10}} + ... \cong 0.9dB
    \end{equation}

\section{Components List}
    The components list is attached in the end of the report.

\section{Post-Test Results}
\subsection{LNA}
    \begin{figure}[t!]
        \begin{center}
            \vspace{-0.3in}
            \includegraphics[width=6in]{../../img/pre_order/lna1.jpg}
            \caption{LNA Test Connection}
            \label{fig:lna1}
        \end{center}
    \end{figure}

    \begin{figure}[b!]
        \begin{center}
            \vspace{-0in}
            \includegraphics[width=6in]{../../img/pre_order/lna2.jpg}
            \caption{LNA Test Results}
            \label{fig:lna2}
        \end{center}
    \end{figure}

    Figure~\ref{fig:lna1} shows how we test the cascading LNAs. The signal that
    goes into the input of the system, which is the antenna input. The input
    power is $0dbm$. The output of the cascading LNAs is in Figure~\ref{fig:lna2}.
    The output signal magnitude is $-35.94dbm$. Thus the total cascading LNAs
    do not provide any gain, while it has a lot of loss. In the demo of our
    system, we also shows that if the input signal is $16dbm$, the output of
    the LNAs is $-9dbM$. Considering that the cables may have $5dB$ loss,
    and the entire system will have about $20dB$ loss.

    Since we turn of the supply signal, there is no output signal at all. Thus I
    think the LNAs are still working, but the matching networks do not matched
    exactly to $50\Omega$. Thus there is a loss around $20dB$ to $30dB$ loss.
    We test the $S11$ and $S21$ on the network analyzer. $S11$ shows that $90\%$
    of the signal has been reflected back. And $S21$ proves the the output of
    spectrum analyzer. 

\subsection{Mixer}
    \begin{figure}[t!]
        \begin{center}
            \vspace{-0.3in}
            \includegraphics[width=6in]{../../img/pre_order/mixer1.jpg}
            \caption{Mixer Test Connection}
            \label{fig:mixer1}
        \end{center}
    \end{figure}

    \begin{figure}[b!]
        \begin{center}
            \vspace{-0in}
            \includegraphics[width=6in]{../../img/pre_order/mixer2.jpg}
            \caption{Mixer Test Results}
            \label{fig:mixer2}
        \end{center}
    \end{figure}

    Figure~\ref{fig:mixer1} shows how we test our Mixer. We use a splitter to
    split the signal from the function generator. The splitter generates two
    $1GHz$ signals as the modulator's local oscillator and radio frequency
    signal for the testing purpose. As we can see from Figure~\ref{fig:mixer2},
    the results show that there is no output at $2GHz$. Thus there should be a
    problem with the \textit{mixer}.

    The possible problem can be soldering problem due to the size of the mixer
    is too small. We check the \textit{VCC} and \textit{GND}, and they are
    connected correctly. The only problem can be the false connection between
    the signal pins.

\subsection{VCO}
    \begin{figure}[t!]
        \begin{center}
            \vspace{-0.3in}
            \includegraphics[width=6in]{../../img/pre_order/900vco1.jpg}
            \caption{$900MHz$ VCO Test Connection}
            \label{fig:900vco1}
        \end{center}
    \end{figure}

    \begin{figure}[b!]
        \begin{center}
            \vspace{-0in}
            \includegraphics[width=6in]{../../img/pre_order/900vco2.jpg}
            \caption{$120MHz$ VCO Test Connection}
            \label{fig:120vco1}
        \end{center}
    \end{figure}

    \begin{figure}[t!]
        \begin{center}
            \vspace{-0.3in}
            \includegraphics[width=6in]{../../img/pre_order/120vco1.jpg}
            \caption{$120MHz$ VCO Test Results}
            \label{fig:120vco2}
        \end{center}
    \end{figure}

    \begin{figure}[b!]
        \begin{center}
            \vspace{-0in}
            \includegraphics[width=6in]{../../img/pre_order/120vco2.jpg}
            \caption{$900MHz$ VCO Test Results}
            \label{fig:900vco2}
        \end{center}
    \end{figure}

    In Figure~\ref{fig:900vco1} and Figure~\ref{fig:120vco1}, we show how we
    connecting the testing for VCO. The VCO results are in
    Figure~\ref{fig:900vco2} and Figure~\ref{fig:120vco2}.
    Figure~\ref{fig:900vco2} shows that the $900MHz$ VCO is not working and
    Figure~\ref{fig:120vco2} shows that the $120MHz$ VCO has a lot more
    harmonics than it should have.

\section{Summary}
    The entire design with the PCB layout design mainly targets at module specs
    managements, which basicly designs for trade-off between different
    components and modification between different modules. The most critical
    part of the design is to have a good link budget estimation and the
    arrangement in the layout design. 

    However, during the testing process, most of the modules do not work well.
    Part of the problem can the parts do not work well. For the LNAs, we did not
    use the third stage. Instead, we used a wire directly bypassing the third
    stage, which may introduce a lot of inductance at the output. This
    eventually leads to the loss to the whole system. For some other problems,
    they may be due to the soldering problem. I think if we have another chance,
    we can try it for the other time.

\end{document}

